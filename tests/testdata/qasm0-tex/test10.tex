% 
% File:	  test10.qasm
% Date:	  22-Mar-04
% Author: I. Chuang <ichuang@mit.edu>
%
% Sample qasm input file - multi-qubit gates
% also demonstrates use of classical bits 
% 
% 	qubit	q0	
% 	cbit	c1
% 	qubit	q2
% 	
% 	h	q0
% 	Utwo	q0,c1
% 	S	q2
% 	Utwo	c1,q2

%  Time 01:
%    Gate 00 h(q0)
%    Gate 02 S(q2)
%  Time 02:
%    Gate 01 Utwo(q0,c1)
%  Time 03:
%    Gate 03 Utwo(c1,q2)

% Qubit circuit matrix:
%
% q0: gAxA, gBxA, n  , n  
% c1: N  , gBxB, gCxB, N  
% q2: gAxC, n  , gCxC, n  

\documentclass[preview]{standalone}
\input{xyqcirc.tex}

% definitions for the circuit elements

\def\gAxA{\op{H}\w\A{gAxA}}
\def\gBxA{\gnqubit{U}{d}\w\A{gBxA}}
\def\gBxB{\gspace{U}\W\A{gBxB}}
\def\gAxC{\op{S}\w\A{gAxC}}
\def\gCxB{\gnqubit{U}{d}\W\A{gCxB}}
\def\gCxC{\gspace{U}\w\A{gCxC}}

% definitions for bit labels and initial states

\def\bA{ \q{q_{0}}}
\def\bB{   {c_{1}}}
\def\bC{ \q{q_{2}}}

% The quantum circuit as an xymatrix

\xymatrix@R=5pt@C=10pt{
    \bA & \gAxA &\gBxA &\n   &\n  
\\  \bB & \N   &\gBxB &\gCxB &\N  
\\  \bC & \gAxC &\n   &\gCxC &\n  
%
% Vertical lines and other post-xymatrix latex
%
}

\end{document}
